\documentclass[a4paper]{scrartcl}
%\usepackage{booktabs}

\newcommand*{\robfilter}{\texttt{robfilter} }
\addtolength{\voffset}{-1cm}
\addtolength{\textheight}{2cm}
%===================================================================
\title{Getting started with \robfilter}
\author{Matthias Borowski, Roland Fried, and Karen Schettlinger}
\date{}
\begin{document}
\maketitle

%===================================================================
\section{Introduction}

\robfilter is a package of functions for robust extraction of an underlying signal from time series.
The general idea is to approximate the data in a moving time
window by a constant level (location-based methods) or a linear
trend (regression-based methods). The several functions differ
with respect to the signal characteristics and the outlier
patterns they can deal with. Some of the procedures are available
both for delayed filtering by approximating the signal in the
window centre and for full online analysis by estimating the
signal value at the end of the time window. The former versions
generally lead to better approximations, but for the costs of a
time delay corresponding to half a window width.

%===================================================================
\section{Overview}

This package contains the following functions: \vspace*{0.25cm} \\
\begin{tabular}{ll}
\texttt{robreg.filter} & simple regression filters (delayed and online) \\
\texttt{dw.filter}     & location or regression based filters working in two steps with \\
                       & possibly different window widths (delayed and online)\\
\texttt{hybrid.filter} & median and repeated median hybrid filters (delayed)\\
\texttt{robust.filter} & regression filter with additional rules for outlier replacement \\
                       & and level shift tracing (delayed and online)\\
\texttt{adore.filter}  & repeated median filter with adaptive choice of the window width\\
                       & (online)\\
\texttt{madore.filter} & multivariate repeated median-least squares filter with adaptive\\
											 & choice of the window width (online)\\
\texttt{wrm.filter}    & weighted repeated median filters (delayed and online)\\
\texttt{wrm.smooth}    & weighted repeated median smoother (delayed and online)
\end{tabular} \vspace*{0.25cm}

\noindent
A more detailed explanation of the single functions can be found
below.

\newpage
%===================================================================
\subsection{\texttt{robreg.filter}}

Running medians are the classical window-based robust filtering
procedures. They simply calculate the median of the data in each
window for the filter output. This leads to good results only when
the signal is approximately constant within each time window, and
it generally leads to delayed estimates. Running medians should
only be applied using very short time windows in case of time
trends. Better results % a better approximation of locally linear trends.
can be achieved by replacing the median by a robust regression
method.

The \texttt{robreg.filter} function applies  simple robust linear
regression for fitting a straight line to the data within each time
window and estimates the current level of the time series either by
the fitted value in the window centre (\texttt{online=FALSE} where
\texttt{width} must be odd) or by the fitted value at the recent
most time point within the time window (option
\texttt{online=TRUE}).

Possible options for the robust regression (\texttt{method})
applied within one window are: the repeated median of Siegel
(1982) (\texttt{"RM"}), least median of squares (\texttt{"LMS"};
Hampel, 1975; Rousseeuw, 1984), least trimmed squares
(\texttt{"LTS"}; Rousseeuw, 1983), least quartile differences
(\texttt{"LQD"}; Croux, Rousseeuw and H\"{o}ssjer, 1994) and deepest
regression (\texttt{"DR"}; Rousseeuw and Hubert, 1999). For
comparison, the simple running median can also be calculated
setting \texttt{method="MED"}.

Comparisons of the performance of certain of the above named
methods can be found in Fried and Gather (2002) and Davies, Fried
and Gather (2004)  for the delayed filter versions, and in Gather,
Schettlinger and Fried (2006) for the online versions. Fast
algorithms for the running \texttt{RM} and \texttt{LQD} are
described by Bernholt and Fried (2003) and Bernholt, Nunkesser and
Schettlinger (2007), respectively.

%===================================================================
\subsection{\texttt{dw.filter}}

Double window filters improve on simple regression-based filters
at the occurrence of level shifts. They are based on the idea of
trimming all observations deviating too much from a rough initial
fit obtained from a possibly shorter inner time window (Bernholt,
Fried, Gather, Wegener, 2006). Only the observations which have
not been trimmed in this preliminary step are included in the
calculation of the final filter output. Both location-based and
regression-based procedures have been implemented, as well as full
online and delayed versions.


%===================================================================
\subsection{\texttt{hybrid.filter}}

Repeated median hybrid filters further improve the preservation of
level shifts and local extremes of the signal value ('turning
points') by calculating several subfilters from different parts of
the window (Fried, Bernholt, Gather, 2006). The price of this is
reduced smoothing and robustness. Different versions of this
approach have been designed for locally constant or locally linear
signals, all versions working with some delay.

%===================================================================
\subsection{\texttt{robust.filter}}

The function \texttt{robust.filter} uses rules for outlier and
shift detection to improve the results of an ordinary repeated
median filter (Fried, 2004). For this, a robust scale estimator
and suitable multiples of the estimated standard deviation need to
be chosen. Additionally, the window width can be allowed to be
time-varying so that the procedure adapts itself to time-varying
slopes and local signal characteristics. The adaption of the
window width is described by Gather and Fried (2004). For more
explanations on shift detection, see Fried and Gather (2007).

%===================================================================
\subsection{\texttt{adore.filter}}

The \texttt{adore.filter} overcomes the problem of window width choice and the accompanying bias variance trade-off for the signal level estimations by choosing the window width adaptively, depending on the current data situation. The filter produces online signal estimates (and if wanted also an accompanying robust online scale estimates) which do not require any prior parameter specification. Therefore, it provides a simple alternative to online filters applying rules for shift or change detection but require the specification of certain settings. For more details see Schettlinger, Fried and Gather (2008).

%===================================================================
\subsection{\texttt{madore.filter}}

The \texttt{madore.filter} is an adaptive online filter for multivariate time series, based on repeated median and multivariate least squares regression. The test procedure of the \texttt{adore.filter} (see above) is used to adapt the window width at each time point, depending on the current data situation. The signal is estimated within the adapted time window by a slight modification of the multivariate \textit{Trimmed Repeated Median-Least Squares} regression (Lanius, Gather, 2004). In contrast to the \texttt{adore.filter}, the \texttt{madore.filter} takes account of the local correlations between the components of the multivariate time series. A more detailed description of the filter can be found in Borowski, Schettlinger, Gather (2008).
%===================================================================
\subsection{\texttt{wrm.filter}}

Weighted repeated median filters weight the observations in the
time window according to their temporal distance to the current
target point at which the signal is estimated (Fried, Einbeck,
Gather, 2007). This weighting allows to increase the window width as compared to
repeated median filters, and thus it increases the
efficiency of noise reduction without increasing the bias at level
shifts. They can be particularly recommended for robust
detail-preserving signal extraction without time delay.

%===================================================================
\subsection{\texttt{wrm.smooth}}

Weighted repeated median smoothing can be seen as a generalisation
of weighted repeated median filtering allowing for non-equidistant
time points. More generally, it can be applied for robust
non-parametric regression of a dependent variable $y$ on an
independent (regressor) variable $x$ and preserves discontinuities
in the regression function. Accordingly, this function needs two
variables, a dependent and an independent one.




\vspace{1.5cm}
\noindent
A comparison of some of the above named methods can be found in
Gather and Fried (2004) (including some regression filters and
some double window and hybrid filters) and an overview over all
methods is given in Schettlinger, Fried and Gather (2006).

%===================================================================
\section{Contact}

%In case of questions, remarks, suggestions, please contact

Matthias Borowski\\
Department of Statistics, Dortmund University of Technology, 44221 Dortmund, Germany\\
E-Mail: \texttt{borowski@statistik.tu-dortmund.de}\vspace*{0.5cm}\\
Roland Fried\\
Department of Statistics, Dortmund University of Technology, 44221 Dortmund, Germany\\
E-Mail: \texttt{fried@statistik.tu-dortmund.de}\vspace*{0.5cm}\\
Karen Schettlinger\\
Department of Statistics, Dortmund University of Technology, 44221 Dortmund, Germany\\
E-Mail: \texttt{schettlinger@statistik.tu-dortmund.de}

%===================================================================
\section{References}

\renewcommand{\labelitemi}{ }
\begin{itemize}
\item
Bernholt, T., Fried, R. (2003) Computing the Update of the Repeated
Median Regression Line in Linear Time, \emph{Information Processing
Letters} \textbf{88}(3), 111-117.

\item
Bernholt, T., Fried, R., Gather, U., Wegener, I. (2006) Modified
Repeated Median Filters, \emph{Statistics and Computing} \textbf{16}, 177-192.

\item
Bernholt, T., Nunkesser, R., Schettlinger, K. (2007) Computing
the Least Quartile Difference Estimator in the Plane,
\emph{Computational Statistics and Data Analysis, Special Issue on
Statistical Algorithms and Software}, to appear.\\
(earlier version:
Technical Report 51/2005, SFB 475, Universit\"{a}t Dortmund)

\item
Borowski, M., Schettlinger, K., Gather, U. (2008) Multivariate Real Time
Signal Estraction by a Robust Adaptive Regression Filter, submitted.

\item
Croux, C. and Rousseeuw, P.J., H\"{o}ssjer, O.
(1994) Generalized S-Estimators, \emph{Journal of the American Statistical
Association}
\textbf{89}, 1271-1281.

\item
Davies, P.L., Fried, R., Gather, U. (2004) Robust Signal Extraction
for On-line Monitoring Data, \emph{Journal of Statistical Planning
and Inference} \textbf{122}, 65-78.

\item
Fried R. (2004) Robust Filtering of Time Series with Trends, \emph{Journal
of Nonparametric Statistics} \textbf{16}(3), 313-328.

\item
Fried, R., Bernholt, T., Gather, U. (2006) Repeated Median and
Hybrid Filters, \emph{Computational Statistics and Data Analysis} \textbf{50},
Special Issue on 'Statistical Signal Extraction and Filtering',
2313-2338.

\item
Fried, R., Einbeck, J., Gather, U. (2007) Weighted Repeated Median
Smoothing and Filtering, \emph{Journal of the American Statistical
Association}, to appear.\\
(earlier version:
Technical Report 33/2005, SFB 475, Universit\"{a}t Dortmund)

\item
Fried, R., Gather, U. (2002) Fast and Robust Filtering of Time Series
with Trends, in: W. H\"{a}rdle and B. R\"{o}nz (eds.),
\emph{Proceedings in Computational Statistics}, Physica, Heidelberg, 367-372.

\item
Fried, R., Gather, U. (2007), On Rank Tests for Shift
Detection in Time Series, \emph{Computational Statistics and Data
Analysis, Special Issue on Machine Learning and Robust Data
Mining} \textbf{52}, 221-233.

\item Gather, U., Fried, R. (2004) Methods and Algorithms for Robust
Filtering, in: J. Antoch (ed.), \emph{Proceedings in Computational
Statistics}, Physica, Heidelberg,  159-170.

\item Gather, U., Schettlinger, K., Fried, R. (2006) Online Signal
Extraction by Robust Linear Regression, \emph{Computational
Statistics}, \textbf{21}(1), 33-51.

\item
Hampel, F.R. (1975) Beyond Location Parameters: Robust Concepts
and Methods, \emph{Bulletin of the International Statistical
Institute} \textbf{46}, 375-382.

\item
Lanius, V., Gather, U. (2007) Robust Online Signal Extraction from
Multivariate Time Series, \emph{Technical Report 38/2007, SFB 475,
Technische Universit\"{a}t Dortmund.}

\item
Rousseeuw, P.J.(1983) Multivariate Estimation with High Breakdown
Point, \emph{Proceedings of the 4th Pannonian Symposium on Mathematical
Statistics and Probability} \textbf{B}, Dordrecht, Reidel.

\item Rousseeuw, P.J. (1984) Least Median of Squares Regression,
\emph{Journal of the American Statistical Association} \textbf{79}, 871-880.

\item
Rousseeuw, P.J., Hubert, M. (1999) Regression Depth, \emph{Journal of the
American Statistical Association} \textbf{94}, 388-402.

\item
Schettlinger, K., Fried, R., Gather, U. (2006), Robust Filters
for Intensive Care Monitoring - Beyond the Running Median,
\emph{Biomedizinische Technik} \textbf{51}(2), 49-56.

\item
Schettlinger, K., Fried, R., Gather, U. (2008),
Real Time Signal Processing by Adaptive Repeated Median Filters,
\emph{International Journal of Adaptive Control and Signal Processing}, submitted.

\item
Siegel, A.F. (1982) Robust Regression Using Repeated
Medians, \emph{Biometrika} \textbf{69}, 242-244.
\end{itemize}
%===================================================================
\end{document}
%===================================================================









Fried, R.: Robust Location Estimation under Dependence, to appear
in: Journal of Statistical Computation and Simulation (2006).

Fried, R., Gather, U.: Robust Trend Estimation for AR(1)
Disturbances, Austrian Journal of Sta-tistics, Vol. 34 (2) (2005),
pp. 139-151.

Gather, U., Fried, R.: Robust Estimation of Scale for Local Linear
Temporal Trends, Tatra Mountains Mathematical Publications 26
(2003), pp. 87-101.

Imhoff, M., Bauer, M., Gather, U., Fried, R.: Pattern Detection in
Intensive Care Monitoring Time Series with Autoregressive Models:
Influ-ence of the Model Order, Biometrical Journal 44-6 (2002),
pp. 746-761.


















Tukey, J.W.: Exploratory Data Analysis, Reading, MA,
Addison-Wesley, 1977.

Heinonen, P., Neuvo, Y.: FIR-Median Hybrid Filters, IEEE
Transactions on Acoustics, Speech, and Signal Processing 35
(1987), pp. 832-838.

Heinonen, P., Neuvo, Y.: FIR-Median Hybrid Filters with Predictive
FIR Substructures, IEEE Transactions on Acoustics, Speech, and
Signal Processing 36 (1988), pp. 892-899.

Heinonen, P., Kalli, S., Turjanmaa, V., Neuvo, Y.: Generalized
Median Filters for Biological Sig-nal Processing, Proceedings of
the Seventh European Conference on Circuit Theory and Design
(1985), pp. 283-386.

Lee, Y., Kassam, S.A.: Generalized Median Fil-tering and Related
Nonlinear Filtering Tech-niques, IEEE Transactions on Acoustics,
Speech and Signal Processing 33 (1985), pp. 672-683.

Huber, P.J.: Robust Statistics, Wiley, New York, 1981.

Himayat, N., Kassam, S.A.: Approximate Per-formance Analysis of
Edge Preserving Filters, IEEE Transactions on Signal Processing 4
(1993), pp. 2764-2776.

Rousseeuw, P.J., Croux, C.: Alternatives to the Median Absolute
Deviation, Journal of the American Statistical Association 88
(1993), pp. 1273-1283.

Gr\"{u}bel, R.: Length of the Shorth, Annals of Sta-tistics 16 (1988),
pp. 619-628.

Rousseeuw, P.J., Leroy, A.: A Robust Scale Es-timator Based on the
Shortest Half, Statistica Neerlandica 42 (1988), pp. 103-116.
